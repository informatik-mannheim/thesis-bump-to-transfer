\chapter{Zusammenfassung und Ausblick}
\label{Kap6}
\label{chap:Kap6}

Das zu Beginn dieser Arbeit formulierte Ziel wurde durch die folgenden Teilschritte erreicht: Zu Beginn wurde beschrieben, in welchen Varianten die Bump-Geste durchgeführt werden kann und welche Kategorien von Endgeräten für die Geste geeignet sind. Es wurden anschließend Technologien identifiziert, die eine Umsetzung der Geste auf mobilen Endgeräten ermöglicht. Für die Erkennung von Bumps wurde untersucht welche Daten von einem Beschleunigungssensor durch einen Bump erzeugt werden. Basierend auf diesen Erkenntnissen wurde ein Algorithmus entwickelt der Bump charakteristische Muster in Sensordaten erkennt. Für die Datenübertragung wurde ein Konzept entwickelt, mit dem Daten, entweder über vorhandene WLAN-Netzwerke oder durch die Bildung von AD-Hoc Netzen, ausgetauscht werden. Des weiteren wurde ein Konzept entwickelt, in dem Daten zwischen den Endgeräten über einen Server ausgetauscht werden können. Damit der Datenaustausch zwischen den richtigen Geräten erfolgt, wurden zwei Konzepte entwickelt mit denen Endgeräte in lokalen Netzwerken und auf Servern identifiziert werden können. Diese funktionieren zum einen über die Erfassung und den Vergleich von Timestamp und GPS und zum anderen über die Identifizierung über iBeacon GeräteIDs. Als praktischer Teil der Arbeit wurde eine Applikation auf der iOS Plattform implementiert mit der sich Endgeräte über iBeacon identifizieren und über lokale WLAN Netzwerke oder AD-Hoc Netzwerke Bilddateien austauschen können.

Aufbauend auf den Ergebnissen dieser Arbeit sind einige weiterführende Arbeiten denkbar. Es wäre interessant, auch eine Applikation auf der Android Plattform zu implementieren und dort zu prüfen, ob auch \ac{NFC} anstatt iBeacon zur Identifizierung der Partnergeräte genutzt werden kann. Ebenfalls von Interesse wäre eine zusätzliche Implementierung eines Server basierten Systems um die Leistungsfähigkeit der Systeme vergleichen zu können.

Weitere Arbeiten zur Usability der Interaktion sind ebenfalls möglich. Es muss noch erforscht werden wie stark ein Bump sein muss, damit die Interaktion für den Anwender ein angenehmes Nutzergefühl besitzt. In diesem Zusammenhang sollte der Algorithmus zur Bump-Erkennung auch auf seine Zuverlässigkeit überprüft und optimiert werden. Gerade bei Bewegungen die Bump ähnliche Muster erzeugen muss die Erkennung noch optimiert werden, da dort häufig fälschlicherweise Bumps erkannt werden. Es stellt sich auch die Frage ob die Bump-Interaktion für Anwender im Kontext der Datenübertagung überhaupt intuitiv ist und welche Anwendungskontexte generell vom Anwender mit der Interaktion assoziiert werden. So könnte die Interaktion aus Anwendersicht eventuell besser geeignet sein um die Displays mobiler Endgeräte durch das Zusammenstoßen aneinander zu klonen oder zu erweitern.