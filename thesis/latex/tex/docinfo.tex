% -------------------------------------------------------
% Daten für die Arbeit
% Wenn hier alles korrekt eingetragen wurde, wird das Titelblatt
% automatisch generiert. D.h. die Datei titelblatt.tex muss nicht mehr
% angepasst werden.

\newcommand{\hsmasprache}{de} % de oder en für Deutsch oder Englisch

% Titel der Arbeit auf Deutsch
\newcommand{\hsmatitelde}{Implementierung einer gestenbasierten Interaktion zum Datenaustausch zwischen mobilen Endgeräten}

% Titel der Arbeit auf Englisch
\newcommand{\hsmatitelen}{Implementation of a Gesture Based Interaction to Transfer Data Between Mobile Devices}

% Weitere Informationen zur Arbeit
\newcommand{\hsmaort}{Mannheim}    % Ort
\newcommand{\hsmaautorvname}{Benjamin} % Vorname(n)
\newcommand{\hsmaautornname}{Grab} % Nachname(n)
\newcommand{\hsmadatum}{04.03.2015} % Datum der Abgabe
\newcommand{\hsmajahr}{2015} % Jahr der Abgabe
\newcommand{\hsmafirma}{Paukenschlag GmbH, Mannheim} % Firma bei der die Arbeit durchgeführt wurde
\newcommand{\hsmabetreuer}{Prof. Kirstin Kohler, Hochschule Mannheim} % Betreuer an der Hochschule
\newcommand{\hsmazweitkorrektor}{Horst Schneider, B. Sc, Hochschule Mannheim} % Betreuer im Unternehmen oder Zweitkorrektor
\newcommand{\hsmafakultaet}{I} % I für Informatik
\newcommand{\hsmastudiengang}{IB} % IB IMB UIB IM MTB
  
% -------------------------------------------------------
% Abstract

% Kurze (maximal halbseitige) Beschreibung, worum es in der Arbeit geht auf Deutsch
\newcommand{\hsmaabstractde}{Anwender benutzen eine Vielzahl von mobilen Geräten, wie Smartphones und Tablets, um ihren täglichen Aktivitäten nachzugehen. Dabei wechseln Sie täglich mehrmals zwischen ihren Geräten, abhängig von Tätigkeit und Kontext. Bei diesem Wechsel wird jedoch der Fortschritt, der durchgeführten Tätigkeit, vom Ursprungsgerät nicht auf das Folgegerät übertragen. Der Anwender muss manuelle Maßnahmen zum Datentransfer zwischen seinen Endgeräten durchführen, die ihn von seiner eigentlichen Tätigkeit abhalten. Gestensteuerung bietet großes Potenzial, eine einfache Interaktion für den Anwender zu bieten, mit der Daten übertragen werden können. Diese Arbeit untersucht deshalb die gestenbasierte Bump-Interaktion, bei der durch einfaches Anstoßen eines Endgeräts Daten auf ein anderes zur Synchronisierung übertragen werden. Im Rahmen dieser Arbeit wird diese Geste analysiert, verschiedene Konzepte für eine Umsetzung erarbeitet, Vor- und Nachteile der Entwürfe dargestellt und einer der Entwürfe prototypisch umgesetzt.}

% Kurze (maximal halbseitige) Beschreibung, worum es in der Arbeit geht auf Englisch

\newcommand{\hsmaabstracten}{On a daily basis people use several mobile devices, like Smartphones or Tablets, to fullfil all kinds of tasks. While doing so they often switch from one devices to another to leverage all the advantages of their device like screen size. When the user switches his device while performing some task he has to manually transfer data to the new device, to continue the task, he has begun on the previous device. For the user this manual data transfer is tedious and keeps him from doing, what he actually wants to do. He just wants to continue, where he left off, when he switches devices. Gesture based interactions can be used, to solve this problem. With the bump-gesture there is already a known interaction which can be used to transfer data between devices. This thesis dicusses design and development of the bump gesture. It analyses the bump-gesture, shows several system designs, discusses advantages and disadvantages of the desings and implements a prototype Application based on one of the designs.
 }



