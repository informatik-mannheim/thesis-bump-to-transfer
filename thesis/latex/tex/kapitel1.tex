 \chapter{Einleitung}
\label{Kap1}
\label{chap:Kap1}

\section{Problemstellung}
Immer mehr Anwender nutzen nicht mehr exklusiv den Desktop PC, sondern eine Vielzahl an Endgeräten um ihren täglichen Aktivitäten nachzugehen. Abhängig vom Kontext, das heißt von Ziel, Aufenthaltsort und Zeitpunkt, entscheiden Anwender, welche Geräte sich für eine Situation am besten eignen und treffen entsprechend eine Auswahl \cite{Google2014:Online}. Dabei kann es vorkommen, dass zur Erledigung einer Aufgabe, mehrere Geräte gleichzeitig oder aufeinanderfolgend genutzt werden. Dadurch lassen sich die Vorteile der verschiedenen Geräte, wie Bildschirmgröße, Mobilität oder Interaktionsmöglichkeiten ausnutzen. Nach einer Studie von Google ist dieses multi-screening bei einem Großteil der Anwender zu beobachten. So beginnen beispielsweise 90\% der in der Studie befragten Personen eine Aktivität auf einem Gerät und führen diese zu einem späteren Zeitpunkt auf einem anderen Gerät fort \cite{Google2014:Online}. Für den Anwender ist diese vermischte Nutzung von Endgeräten somit ein alltäglicher Vorgang. Damit ein Anwender seine Aktivität bei einem Gerätewechsel fortsetzen kann, müssen häufig erst Daten zwischen den Geräten ausgetauscht werden. Dieser Datenaustausch geschieht häufig nicht automatisch und erfordern vom Anwender eine Reihe von komplizierten Interaktionen zum Datentransfer zwischen den Geräten. Für den Anwender stellt dies eine Störung seines Arbeitsablaufes dar. Er erwartet fließend zwischen Geräten wechseln zu können da er seine Aktivität über Gerätegrenzen hinaus als zusammengehörig empfindet.

Mit Gestensteuerung können Interaktionen verwirklicht werden, die über Gerätegrenzen hinaus wirken. Somit kann der für den Anwender aufwendige  Interaktionen zur Datenübertragung beim Gerätewechsel durch eine einzelne intuitive Interaktion durchgeführt werden. Ein vielversprechender Kandidat für solch eine Interaktion ist die Bump-Geste. Die Bump-Geste ist eine Interaktion, bei der Daten zwischen Endgeräten ausgetauscht werden indem diese sich gegenseitig anstoßen (bumpen). Anwender, die ihr Gerät wechseln, wollen mit dem Gerät, auf dem die Aktivität begonnen wurde, das Folgegerät bumpen und damit alle notwendigen Daten übertragen, um ihre Aktivität sofort auf dem neuen Gerät weiterzuführen zu können. Betrachtet ein Anwender z.B. ein Photoalbum auf seinem Smartphone und möchte diese Aktivität auf seinem Tablet fortsetzen, muss er nur das Smartphone an das Tablet bumpen. Anschließend wird das Photoalbum auf das Tablet übertragen und die Applikation navigiert automatisch zu Photoalbum und Bild, bei dem die Aktivität auf dem Smartphone beendet wurde.

\section{Ziel der Arbeit}
Das Ziel dieser Arbeit ist es, verschiedene Systementwürfe zu identifizieren und zu beschreiben, mit denen die Bump Interaktion realisiert werden kann. Diese Entwürfe berücksichtigen unterschiedliche technische Rahmenbedingungen und werden auf ihre Vor- und Nachteile geprüft. und Aufbauend auf einem der identifizierten Systementwürfe wird eine Demonstrator Applikation entwickelt mit der ein aktives Dokument oder Foto auf ein anderes Gerät übertragen werden kann.

\section{Aufbau der Arbeit}
In Kapitel \ref{chap:Kap2} werden verwandte Arbeiten und Applikationen zum Thema sowie Technische Grundlagen beschrieben. Kapitel \ref{chap:Kap3} analysiert und beschreibt die Bump-Interaktion. Kapitel \ref{chap:Kap4} beschreibt einleitend Grundlagen zu relevanten iOS Frameworks auf deren Basis  Konzepte zur Bump Erkennung, Geräte Identifizierung und zum Datenaustausch beschrieben und verglichen werden. In Kapitel \ref{chap:Kap5} wird die entwickelte Demonstrator Applikation vorgestellt. Abschließend bietet Kapitel \ref{chap:Kap6} eine Zusammenfassung über die erarbeiteten Ergebnisse und einen Ausblick auf zukünftige potenzielle Weiterentwicklungen der Arbeit. Die Bedeutungen der verwendeten Abkürzungen befinden sich im \nameref{Abk} und ein Überblick der dargestellten Abbildungen im Abbildungsverzeichnis sowie der Tabellen im Tabellenverzeichnis. Zudem ist die verwendete Literatur im Literaturverzeichnis angegeben.